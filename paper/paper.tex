\documentclass[runningheads,a4paper]{llncs}

\usepackage{amssymb}
\setcounter{tocdepth}{3}
\usepackage{graphicx}
\usepackage{pgfplots}

\usepackage{url}
\newcommand{\keywords}[1]{\par\addvspace\baselineskip
\noindent\keywordname\enspace\ignorespaces#1}

\begin{document}

\mainmatter

\title{Fast Nearest Neighbour Classification}
\subtitle{\textnormal{\small{Seminar\\
Intelligente Software-Systeme\\
Sommersemester 2015\\\vspace{1\baselineskip}
Stefan Fricke\\\vspace{2\baselineskip}
April 2016\\
Betreuer: Stephan Spiegel\\\vspace{1\baselineskip}}}}

\titlerunning{Fast Nearest Neighbour Classification}

\author{Gordon Lesti\\313249\\Studiengang: Informatik\\gordon.lesti@campus.tu-berlin.de\\\vspace{5\baselineskip}}

\authorrunning{Fast Nearest Neighbour Classification}

\institute{Technische Universit\"at Berlin\\
Fakult\"at IV Elektrotechnik und Informatik\\
Fachgebiet AOT\\
Prof. Dr. Sahin Albayrak\\
\url{http://www.aot.tu-berlin.de/}}

\toctitle{Fast Nearest Neighbour Classification}
\tocauthor{Gordon Lesti}
\maketitle
\pagenumbering{roman}

\addcontentsline{toc}{section}{Abstract}
\begin{abstract}
This term paper is about nearest neighbour classifiers that work with the triangle inequality of metric spaces.
Benchmarks on examples from the two-dimensional space will show the advantages and disadvantages of three known
algorithms. The Orchard’s Algorithm, the Annulus Method and the AESA will be explained on examples and compared to the
full search. The benchmarks aren't comparing execution time, they will compare calls of the distance function.
\keywords{Nearest neighbour classifiers, Triangle inequality, Metric space}
\end{abstract}

\tableofcontents
\addcontentsline{toc}{section}{Table of Contents}

\newpage
\pagenumbering{arabic}

\section{Introduction}

\subsection{Problem}

The problem of the nearest neighbour search is pretty easy described. We have a set of objects and we want to find that
object with the smallest distance to a query object.\\
The requirements are:
\begin{itemize}
	\item a set $\mathbb{U}$
	\item a distance function $d$ on $\mathbb{U}$, with $d: \mathbb{U} \times \mathbb{U} \to \mathbb{R}$
	\item a set $S \subset \mathbb{U}$ with size $n$
	\item a query $q \in \mathbb{U}$
\end{itemize}
The solution is:
\begin{itemize}
	\item an $a \in S$, with $d(q, a) \le d(q, x)$ for all $x \in S$
\end{itemize}

\section{Algorithms}

Every nearest neighbour search algorithm can be divided in two steps.
\begin{itemize}
	\item Preprocessing
	\item Query Processing
\end{itemize}
In the preprocessing the algorithm prepare the given set $\mathbb{S}$ for
their needs.\\
The query processing is the real nearest neighbour search, the algorithms take the query $q$ and return the nearest
neighbour $a$.

\subsection{Full Search}

The full search isn't a real fast nearest neighbour algorithm, it's more the default algorithm for such a problem. We
just walk over all $x \in S$ and calculate $d(q, x)$. After that we just take $a \in S$, so that
$d(q, a) \le d(q, x)$ for all $x \in S$. The big advantage compared to the following algorithms is, that the full search
works also in not metric spaces. The full search needs no preprocessing.

\subsection{Orchard’s Algorithm}

\subsubsection{Preprocessing}

The preprocessing of the Orchard’s Algorithm has a high complexity. For every $p \in S$, we generate a list that
contains all $x \in S\setminus\left\{ {p}\right\}$ with it's distance $d(p, x)$ and order those lists ascending to the
distance. If we assume that the set $S$ has size $n$, the preprocessing will call the distance function
$\frac{n(n-1)}{2}$ times and will sort $n$ lists of size $n-1$.

\subsubsection{Query Processing}

When calling the query processing with query $q$, we randomly select one object $c \in S$ as initial candidate and
calculate $d(c, q)$. After that we walk over the ordered list of $c$ and with every picked $s$ from the list we
calculate $d(l, q)$. If $d(l, q)$ is smaller than $d(c, q)$, $l$ is our new candidate $c$ and we will walk over the
list of $l$. We abort if we reached the end of a list or if $d(c, s) > 2d(c, q)$. The current candidate $c$ is our
nearest neighbour.

\subsection{Annulus Method}

The Annulus Method works a little bit similar to the Orchard’s Algorithm, but with a less complex preprocessing.

\subsubsection{Preprocessing}

Instead of generating a list for all $x \in S$, the Annulus Method generates just one list for a random $p^* \in S$. In
the preprocessing we generate one list that contains all $x \in S$ with it's distance $d(p^*, x)$ and order this list
ascending to the distance. If we assume that the set $S$ has size $n$, the preprocessing will call the distance function
$n$ times and will sort one list of size $n$.

\subsubsection{Query Processing}

The query processing of the Annulus Method starts by picking a random candidate $c \in S$ from the ordered list of
$p^*$. After that we walk alternating away from $p^*$ and back to him in the list. If the current object $s$ has
$d(s, q) < d(c, q)$, we set $s$ as our new candidate $c$.If the current object $s$ is under $c$ in the list and
$d(p^*, s) < d(p^*, q) - d(c, q)$, we can ignore all objects under $s$ in the list. If the current object $s$ is above
$c$ in the list and $d(p^*, s) > d(p^*, q) + d(c, q)$, we can ignore all objects above $s$ in the list. The object $c$
is the nearest neighbour, if the entire list is traversed.

\subsection{AESA}

\subsubsection{Preprocessing}

\subsubsection{Query Processing}

\section{Benchmarks}

As described above, the following benchmarks will not compare the runtime of the mentioned algorithms. The benchmarks
will compare the calls of the distance function that the algorithm needs for preprocessing and for finding a nearest
neighbour. Those benchmarks can make sense when the distance function has a heavy complexity. For example the string
metric Levenshtein distance or in higher dimensional vector spaces.\\
In general those benchmarks are affected by the implementation, by the executing machine, by the domain of the problem
and the distribution of the example.

\subsection{Implementation}

The algorithms above are all implemented in Java and can be found in the following git repository\\
\url{https://github.com/GordonLesti/FastNearestNeighbourClassification}.\\

A nearest neighbour algorithm implemented in Java should handle every type of object that a set $\mathbb{U}$ can
represent under the condition that there is a metric on $\mathbb{U}$. Because of that, a nearest neighbour algorithm
class needs to extend from the abstract class \verb+FastNearestNeighbourClassificator+ that can be found in the package
\verb+fnnc.algo+. The generic \verb+T+ represents an object from the set $\mathbb{U}$ and the generic \verb+D+
represents the distance and has to be of type \verb+Comparable+ from the package \verb+java.lang+.\\

A possible example for the type \verb+T+ can be \verb+Point2D.Double+ from the package
\verb+java.awt.geom+ and a possible way to implement a distance function would be the euclidean distance of type
\verb+Double+.

\subsubsection{FastNearestNeighbourClassificator}

The abstract class\\ \verb+FastNearestNeighbourClassificator<T, D extends Comparable<D>>+ has two protected class
variables:
\begin{itemize}
	\item a \verb+distanceCalculator+ of type \verb+DistanceCalculator<T, D>+
	\item a \verb+objectCollection+ of type \verb+Collection<T>+
\end{itemize}
and three simple class methods:
\begin{itemize}
	\item a constructor that takes an object of type \verb+DistanceCalculator<T, D>+ from the package
		\verb+fnnc.model+ and sets the protected class variable
	\item a default function \verb+preProcessing+ that takes a \verb+Collection<T>+ from the package
		\verb+java.util+ and sets the protected class variable
	\item a abstract function \verb+calculateNearestNeighbour+ that takes the query object of type \verb+T+ and
		returns the nearest neighbour object of type \verb+T+
\end{itemize}

A instance of type \verb+FastNearestNeighbourClassificator+ shouldn't and can't calculate the distance between two
objects. Because of that we inject the \verb+DistanceCalculator+ in the constructor. All implemented algorithms exclude
the \verb+FullSearch+ require that the \verb+DistanceCalculator+ returns a distance of type \verb+Double+ instead of
only \verb+Comparable+.

\subsubsection{DistanceCalculator}

The abstract class\\ \verb+DistanceCalculator<Point2D.Double, Double>+ has just one private \verb+count+ of type
\verb+int+ and three class methods:
\begin{itemize}
	\item a public final function \verb+calculateDistance+ that takes two objects of type \verb+T+, increases the
		internal count and returns the result of \verb+internalCalculateDistance+
	\item a public final function \verb+reset+ that resets the internal count
	\item a protected abstract function \verb+internalCalculateDistance+ that takes two objects of type \verb+T+ and
		returns the distance object of type \verb+D+
\end{itemize}

\subsection{Example data}

The benchmarks will use a \verb+java.util.Collection+ of type\\ \verb+java.awt.geom.Point2D+ that are representing
points in $\mathbb{R}^2$. For every size $n \in \left\{ {100, 200, \dots, 1000}\right\}$, every algorithm will get
thousand sets with randomly generated points. The class \verb+fnnc.model.point2ddouble.DistanceCalculator+ will count
every call of the function \verb+internalCalculateDistance+ and after all we will calculate the average of the calls for
every algorithm.

\subsection{Result}

\subsubsection{Preprocessing}

\begin{figure}
	\begin{center}
		\begin{tikzpicture}
			\begin{axis}[
				ymin=0,
				xmin=0,
				xlabel=size of test data set $S$,
				ylabel=average calls of distance function,
				legend pos=outer north east]
				\addplot table {data/preprocessing/fullsearch.csv};
	            \addlegendentry{Full Search}
	            \addplot table {data/preprocessing/orchards.csv};
	            \addlegendentry{Orchards}
	            \addplot table {data/preprocessing/annulus.csv};
	            \addlegendentry{Annulus}
	            \addplot table {data/preprocessing/aesa.csv};
	            \addlegendentry{AESA}
			\end{axis}
		\end{tikzpicture}
	\end{center}
	\caption{The average calls of the distance function in the preprocessing depending on the size of the test data set
		$S$ visualized with graphs.}
	\label{fig:preprocessing:graph}
\end{figure}

\subsubsection{Query Processing}

\begin{figure}
	\begin{center}
		\begin{tabular}{ | l | c | c | c | c |}
			\hline
		  	& Full Search & Orchard’s Algorithm & Annulus Method & AESA \\ \hline
		  	100 & 100 & 20 & 46 & 4 \\ \hline
			200 & 200 & 26 & 82 & 4 \\ \hline
			300 & 300 & 30 & 122 & 4 \\ \hline
			400 & 400 & 35 & 157 & 4 \\ \hline
			500 & 500 & 39 & 190 & 4 \\ \hline
			600 & 600 & 42 & 222 & 4 \\ \hline
			700 & 700 & 44 & 269 & 4 \\ \hline
			800 & 800 & 46 & 302 & 4 \\ \hline
			900 & 900 & 50 & 334 & 4 \\ \hline
			1000 & 1000 & 52 & 373 & 4 \\ \hline
			1100 & 1100 & 55 & 418 & 4 \\ \hline
			1200 & 1200 & 56 & 433 & 4 \\ \hline
			1300 & 1300 & 59 & 477 & 4 \\ \hline
			1400 & 1400 & 62 & 510 & 4 \\ \hline
			1500 & 1500 & 64 & 544 & 4 \\ \hline
			1600 & 1600 & 63 & 587 & 4 \\ \hline
			1700 & 1700 & 66 & 607 & 4 \\ \hline
			1800 & 1800 & 69 & 667 & 4 \\ \hline
			1900 & 1900 & 72 & 716 & 4 \\ \hline
			2000 & 2000 & 73 & 714 & 4 \\ \hline
		\end{tabular}
	\end{center}
	\caption{The average calls of the distance function in the query processing depending on the size of the test data
		set $S$.}
	\label{fig:queryprocessing:tabular}
\end{figure}

\begin{figure}
	\begin{center}
		\begin{tikzpicture}
			\begin{axis}[ymin=0, xmin=0, legend pos=outer north east]
				\addplot table {data/queryprocessing/fullsearch.csv};
	            \addlegendentry{Full Search}
	            \addplot table {data/queryprocessing/orchards.csv};
	            \addlegendentry{Orchards}
	            \addplot table {data/queryprocessing/annulus.csv};
	            \addlegendentry{Annulus}
	            \addplot table {data/queryprocessing/aesa.csv};
	            \addlegendentry{AESA}
			\end{axis}
		\end{tikzpicture}
	\end{center}
	\caption{The average calls of the distance function in the query processing depending on the size of the test data
		set $S$ visualized with graphs.}
	\label{fig:queryprocessing:graph}
\end{figure}

\end{document}
