\documentclass[deutsch,runningheads,a4paper]{llncs}

\usepackage[utf8]{inputenc}
\usepackage{amsfonts}
\usepackage[T1]{fontenc}
\usepackage[a4paper, left=3cm, right=3cm, top=2.5cm, bottom=2cm]{geometry}

\newcommand{\keywords}[1]{\par\addvspace\baselineskip
\noindent\keywordname\enspace\ignorespaces#1}

\newcommand{\changefont}[3]{
\fontfamily{#1} \fontseries{#2} \fontshape{#3} \selectfont}
\changefont{ptm}{m}{n}
\fontsize{12pt}{6pt}

\begin{document}
  \setcounter{tocdepth}{2}
  \makeatletter
  \renewcommand*\l@author[2]{}
  \renewcommand*\l@title[2]{}
  \makeatletter
  \title{Fast Nearest Neighbour Classification}
  \subtitle{\textnormal{\small{Seminar\\Intelligente Software-Systeme\\Sommersemester 2015}\\\vspace{1\baselineskip} Stefan Fricke\\\vspace{15\baselineskip} Oktober 2015\\Betreuer: Stephan Spiegel\\\vspace{2\baselineskip}}}
  \titlerunning{Fast Nearest Neighbour Classification}
  
  \author{Gordon Lesti\\313249\\Studiengang: Informatik\\gordon.lesti@campus.tu-berlin.de}
  \authorrunning{Gordon Lesti}
  
  \institute{Technische Universität Berlin\\
  Fakultät IV – Elektrotechnik und Informatik\\
  Fachgebiet AOT\\
  Prof. Dr. Sahin Albayrak}
  
  \maketitle
  \newpage
  \pagenumbering{roman}
  
  \addcontentsline{toc}{section}{Zusammenfassung}
  \begin{abstract}
    Diese Seminararbeit befasst sich mit dem Problem der \textit{Nächster Nachbar Klassifikation}. Zur Lösung dieses
    Problems werden grundlegende Algorithmen basierend auf der Dreiecksungleichung in metrischen Räumen vorgestellt und
    verglichen. Dabei hat jeder Algorithmus die Aufgabe aus einer Menge $S$ das Element $a$ zu ermitteln, welches den
    kleinsten Abstand zum Anfragepunkt $q$ hat. $S$ ist dabei eine endliche Teilmenge von $\mathbb{U}$ und $d$ das
    Abstandsmaß auf $\mathbb{U}$.
    \keywords{Nächster Nachbar Klassifikation, Dreiecksungleichung, Metrischer Raum}\\\vspace{4\baselineskip}
    
    \textbf{Plagiatserklärung}\\
    Hiermit versichere ich, dass ich diese Seminararbeit selbstständig verfasst und keine anderen als die angegebenen
    Quellen und Hilfsmittel benutzt habe. Die Stellen meiner Arbeit, die dem Wortlaut oder dem Sinn nach anderen Werken
    entnommen sind, habe ich in jedem Fall unter Angabe der Quelle als Entlehnung kenntlich gemacht. Dasselbe gilt
    sinngemaäß für Tabellen, Karten und Abbildungen. Diese Arbeit hat in dieser oder einer ähnlichen Form noch nicht im
    Rahmen einer anderen Prüfung vorgelegen.
  \end{abstract}
  
  \tableofcontents
  \addcontentsline{toc}{section}{Inhaltsverzeichnis}
  \newpage
  \pagenumbering{arabic}
  
  \section{Einleitung}
    Das Problem der \textit{Nächster Nachbar Klassifikation} lässt sich recht leicht zusammen fassen. Gegeben ist eine
    Menge $\mathbb{U}$ und ein Abstandsmaß $d$ auf $\mathbb{U}$ mit $d: \mathbb{U} \times \mathbb{U} \to \mathbb{R}$.
    Des weiteren eine endliche Teilmenge $S \subset \mathbb{U}$ der Größe $n$ und ein Anfragepunkt $q \in \mathbb{U}$.
    Gesucht ist der nächste Nachbar $a$ aus $S$, welcher minimalen Abstand zu $q$ hat. Also $a \in S$, mit
    $d(q, a) \le d(q, x)$ f\"ur alle $x \in S$.
    
    Eine allgemeinere Form dieses Problems ist die \textit{k-Nächsten Nachbarn Klassifikation}. In dieser
    Semesterarbeit wird somit der Spezialfall mit $k = 1$ betrachtet.
  
  \subsection{Vollsuche}
    Die Vollsuche ist die einfachste Form um einen nächsten Nachbarn $a$ aus der Menge $S$ zu ermitteln.
    Dabei wird für jedes Element $x \in S$ der Abstand $d(q, x)$ berechnet. Der nächste Nachbar zu $q$ ist das Element
    mit dem kleinsten Abstand.
    
    Ein Vorteil der Vollsuche ist, dass die Implementierung sehr einfach ist. Des weiteren funktioniert dieser
    Algorithmus auch in nicht metrischen Räumen. Der große Nachteil ist eindeutig die Laufzeit. Für jedes Element aus
    $S$ muss der Abstand zum Anfragepunkt $q$ berechnet werden.
    
  \section{Algorithmen}
    
  \subsection{Orchard's Algorithmus}
  
  \subsection{Annulus Verfahren}
  
  \subsection{AESA}

\end{document}
